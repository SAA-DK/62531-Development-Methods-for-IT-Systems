\section{Use Case og Diagrammer}

\begin{tcolorbox}[colback=red!10!white,colframe=red!75!black,title=\enumlemma{Use cases} ]
Modeller hvor man gennemgår de forskellige flow der er i en system. Der er \textbf{main flow}, der beskriver hvordan det "perfekte" flow går fra start til slut.\\
Ud fra \textbf{main flowet} kan der så udspringer \textbf{Alternative flow}, som er alle de afvigelser der kan værer fra \textbf{Main flowet}.\\
En Use case har følgende punkter:
\begin{itemize}
    \item Title
    \item ID
    \item Kort beskrivelse af flow
    \item Primary actors
    \item Secondary actors
    \item Main flow/Alternativ flow
    \item Postcondition
    \item Alternative Flows (kun ved Main flow)
\end{itemize}
%\parencite[page 178]{HH} 
\end{tcolorbox}\\

\begin{tcolorbox}[colback=red!10!white,colframe=red!75!black,title=\enumlemma{Use case diagram}]
Et Use case diagram er den simpleste måde at repræsentere en user's interaktion med systemet, der også viser relationen mellem user og de forskellige Use cases hvori at brugeren er involveret.\\
Et Use case diagram kan identificere de forskellige typer af brugere af et system og de forskellige use cases. Et Use case står sjældent alene, og vil oftest blive ledsaget af andre typer af diagrammer.\\
En Use case er representeret enten ved en cirkel eller Eclipse.
%\parencite[page 178]{HH} 
\end{tcolorbox}\\