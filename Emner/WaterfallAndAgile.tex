\section{Waterfall og Agile arbejdsmetoder}

\begin{tcolorbox}[colback=red!10!white,colframe=red!75!black,title=\enumlemma{Waterfall} ]
Waterfall er en arbejdsmetode der kun bør bruges, hvis man enten arbejder med et meget lille projekt, eller man med 100\% ved hvad der skal laves og arbejdes med.\\
Hvis man arbejder med \textbf{waterfall-metoden} har man ikke mulighed for at være "smidig" under arbejdsprocessen. Der er en lige pil fra starten af projektet til slutningen af projektet, og det er ikke muligt at tilfører eller fjerne noget fra arbejdsprocessen. 
%\parencite[page 178]{HH} 
\end{tcolorbox}\\